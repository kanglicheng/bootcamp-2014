\problemname{Palindromes in crosswords}

Mr.~F. just loves palindromes (a palindrome is a word that reads the same
forwards and backwards). He got hold of his mom's solved
crosswords and now he is looking for palindromes in them. The crosswords are somewhat unusual: each is an $n\times n$ grid with no black squares---every location contains a letter. Mr.~F.
considers only sequences of letters that can be read by going
right and down, that is, the next letter in the sequence
is immediately to the right or below the current letter. Help him find the
longest palindrome.

\section*{Input}
The first line contains $k, 1\le k \le 10$, the number of crosswords. Then $k$
crossword descriptions follow. Each crossword description starts
with a number $1\le n\le 100$. Then an $n\times n$ matrix
of uppercase letters follows (there are $n$ letters in each
row; the letters are not separated by spaces and the row ends with a new line).


\section*{Output}
The output contains $k$ lines, one for each crossword. The $i$-th
line contains four items. The first item is one of the longest palindromes that occur in the $i$-th crossword.
The second and third item specify the row and column coordinates of the first letter in the
palindrome, with top left corner at coordinates $(1,1)$.
The fourth item is a string consisting of only letters R, D, and S that indicate how to move through the crossword to get the palindrome:
starting at the specified location, move to the right if the current letter is R or down if the letter is D; stop at letter S, the last letter of the string.


\section*{Note}
A longest palindrome in the first sample input is ABABA. There are several occurrences of this palindrome, the sample output lists the one that starts at location 1,1 and then goes right-right-down-down.

For the second sample input there are only palindromes of length 1. One such palindrome is B, that
can be found at location $(1,2)$.
